\documentclass[11pt,fleqn,twoside]{article}
\usepackage{makeidx}
\makeindex
\usepackage{palatino} %or {times} etc
\usepackage{plain} %bibliography style
\usepackage{amsmath} %math fonts - just in case
\usepackage{amsfonts} %math fonts
\usepackage{amssymb} %math fonts
\usepackage{lastpage} %for footer page numbers
\usepackage{fancyhdr} %header and footer package
\usepackage{mmpv2}
%\usepackage{url}
\usepackage{hyperref}

% the following packages are used for citations - You only need to include one.
%
% Use the cite package if you are using the numeric style (e.g. IEEEannot).
% Use the natbib package if you are using the author-date style (e.g. authordate2annot).
% Only use one of these and comment out the other one.
\usepackage{cite}
%\usepackage{natbib}

\begin{document}

\name{Michal Goly}
\userid{mwg2}
\projecttitle{Quiz Tool}
\projecttitlememoir{Quiz Tool} %same as the project title or abridged version for page header
\reporttitle{Outline Project Specification}
\version{0.1}
\docstatus{Draft}
\modulecode{CS39440}
\degreeschemecode{G600}
\degreeschemename{Software Engineering}
\supervisor{Chris Loftus} % e.g. Neil Taylor
\supervisorid{cwl}
\wordcount{}

%optional - comment out next line to use current date for the document
%\documentdate{10th February 2014}
\mmp

\setcounter{tocdepth}{3} %set required number of level in table of contents

%==============================================================================
\section{Project description}
%==============================================================================
Student engagement and live knowledge monitoring are vital in the provision of good quality
lecture content in 2018. It is important to make sure audience understands concepts presented,
and quizzes can help lecturers judge students' understanding in real time.

Aberystwyth University currently uses Qwizdom\cite{1} live polling tool during the provision
of some lectures and practical sessions. The university operates under a single license
forcing lecturers to book sessions before they can use the tool. Due to human nature,
session hijacking occasionally occurs to the bemusement of both students and lecturers.
For example, students could be shown biology slides half way through their geography
lecture.

This project will focus on the design and development of an in-house built Quiz Tool,
enabling multiple lecturers to use it at the same time and potentially making Qwizdom
redundant in the future. This ambition can only be achieved if the project is of
high quality and its future maintainability is considered at all stages of the design
and development.

The Quiz Tool will allow lecturers to login with their university credentials and
they will be authenticated using the university LDAP. Lecturers will then
be able to upload their PDF lecture slides and create \textit{Lectures} in the system.
Each \textit{Lecture} can be then edited and quizzes can be added between slides.
An example of a quiz would be an "ABC style" question where a student can select
one of the options and submit their answer. Once a lecturer is happy with their
\textit{Lecture}, he can broadcast it and receive a session key which can be
shared with students. Lecture slides will be shown to all students and the
lecture can be delivered in a traditional fashion up to the moment a quiz
has been embedded between two slides. Students will then be able to answer
the question and polling results will be presented in real time to the lecturer.
It will be also possible to export the quiz answers as JSON for future analysis.

The tool will be composed of the back-end with an associated database, and two
front-ends. One for lecturers and one for students. Finally, the tool will be
developed using an agile methodology, adjusted for a single person project.

%==============================================================================
\section{Proposed tasks}
%==============================================================================
The following tasks will be performed on this project:

\begin{enumerate}
  \item \textbf{Choice of the development methodology}: an appropriate methodology has to be
    chosen for a single person project. A variation of SCRUM could be used, with weekly sprints
    and sprint planning/retrospectives shorten to the absolute minimum. This would provide the
    benefits of being able to track the velocity of the project, be able to plan ahead and stay on
    track easier, without completely "hacking" things together on day to day basis.
  \item \textbf{Technology considerations}: the MEAN stack\cite{2} will be used to develop the tool. MEAN
    stack consist of JavaScript based technologies (\textbf{M}ongoDB database, \textbf{E}xpress minimalistic JavaScript web
    development framework, \textbf{A}ngular 4 front-end framework and \textbf{N}odeJS a JavaScript engine).
    Some prototyping will be necessary to gain familiarity with the tools mentioned above. Furthermore,
    each part of the app will be containerised using Docker\cite{3} and containers will run together in the same fashion
    on the developer's machine, the build tool, and in production using docker-compose\cite{4} container orchestration tool.
  \item \textbf{On-site vs external hosting}: it would be beneficial to keep the web application running
    within the university intranet. This however, proved impossible with the LXC debian containers\cite{5} available
    to students for their final year projects. The continuous integration and container deployments mentioned
    in this document will require more customisation and therefore external cloud provider has to be
    chosen to allow suitable deployment.
  \item \textbf{Setting up version control and suitable safeguards}: the code will be stored in a private
    GitHub\cite{6} repository. There will be a \texttt{master} production branch, and a \texttt{development} branch.
    The feature-branch git workflow \cite{7} will be used and each branch will need to have an associated issue (story)
    on GitHub. Then a pull request will be opened between the \texttt{feature-branch} and the \texttt{development}
    and once all tests pass the merge will be allowed. Direct push to both \texttt{master} and \texttt{development}
    will be forbidden.
  \item \textbf{Setting up a continuous integration build agent}: a suitable CI tool needs to be chosen. Travis\cite{8}, Circle CI\cite{9}
    and Jenkins\cite{10} are all appropriate for the task at hand. The build agent will checkout the code from the version control,
    build the Quiz Tool, run all the tests and deploy the tool to production if the branch being built is \texttt{master}.
  \item \textbf{Development}: as previously mentioned, an iterative development approach will be used. Development will start
    with prototyping, followed by the development of "first pass" implementations of all services to get something
    working as soon as possible. Then the tool can be polished to deliver the final product. It is important to quickly
    address the most technically difficult parts of the system to avoid problems later on in the process. Special
    attention needs to be focused on the real time nature of the system and how this should be implemented. Another
    interesting problem will be the persistence of the MongoDB running in a Docker container, as Docker does not
    preserve data by design when containers are destroyed.
  \item \textbf{Project Meetings}: weekly meetings with the supervisor will provide immediate feedback on work done
    and allow re-adjustment of the approach if necessary to stay on track and deliver the software before the deadline.
\end{enumerate}

%==============================================================================
\section{Project deliverables}
%==============================================================================
The following project deliverables are expected.
\begin{enumerate}
  \item \textbf{Stories and burndown charts}: a backlog of issues grouped into epics and milestones will be
    visible on GitHub, together with burndown charts and a Kanban board provided by ZenHub\cite{11}. A summary of these
    will be included in the appendix of the Final Report.
  \item \textbf{Build definition}: a build script will be produced to configure the continuous integration agent.
    Depending on the CI tool used, the configuration file will be stored in the version control.
  \item \textbf{Back-end software}: a Docker containerised web application written in Express, and a MongoDB database
    persistence layer will be developed. The code will be stored in version control, and submitted for assessment
    together with the Final Report.
  \item \textbf{Front-end software}: a Docker containerised Angular 4 front-end will be developed to allow both
    lecturers and students to use the tool. The code will be stored in version control, and submitted for assessment
    together with the Final Report.
  \item \textbf{Development and production environments}: code will run in a similar fashion both locally on the developer's
    machine, and in the production environment provided by a cloud provider. This will be possible due to the docker-compose
    container orchestration tool.
  \item \textbf{Tests}: a set of tests will be added as software will be developed. These will be stored in version
    control alongside the rest of the code.
  \item \textbf{Mid-Project Demonstration PDF presentation}: a PDF presentation will be produced to guide the
    mid-project demonstration.
  \item \textbf{Final Report}: a final document will contain an appendix. The work performed will be
    discussed and there will be acknowledgement of 3rd party libraries, frameworks and tools used.

\end{enumerate}
% the following line is included so that the bibliography is also shown in the table of contents. There is the possibility that this is added to the previous page for the bibliography. To address this, a newline is added so that it appears on the first page for the bibliography.
% \addcontentsline{toc}{section}{Initial Annotated Bibliography}

\begin{thebibliography}{1}

\bibitem{1} Qwizdom, “Qwizdom homepage,” 2018. [Online] Available: https://qwizdom.com/, [Accessed: Feb. 6, 2018].

  Qwizdom is the tool currently used by the univeristy and will be replaced with this project.

\bibitem{2} MEAN stack, [Online] Available: https://github.com/linnovate/mean, [Accessed: Feb. 6, 2018].

  The MEAN stack uses Mongo, Express, Angular(4) and Node for simple and scalable fullstack js applications.

\bibitem{3} Docker, "Docker homepage", [Online] Available: https://www.docker.com/, [Accessed: Feb. 6, 2018].

  Docker allows app to be containerised and run on multiple hosts in the same fashion.

\bibitem{4} docker-compose, "docker-compose homepage", [Online] Available: https://docs.docker.com/compose/, [Accessed: Feb. 6, 2018].

  Compose is a tool for defining and running multi-container Docker applications.

\bibitem{5} LXC, "LXC", [Online] Available: https://wiki.debian.org/LXC, [Accessed: Feb. 6, 2018].

  Linux Containers (LXC) provide a Free Software virtualization system for computers running GNU/Linux. They are inappropriate
  for the task at hand, as they do not allow running docker-compose due to the lack of system permissions.

\bibitem{6} GitHub, Inc., “GitHub homepage,” 2018, [Online] Available: http://github.com/, [Accessed: Feb. 6, 2018].

  A popular version control tool offering private repositories to students. Code will be stored there and checkout by CI for
  testing and deployment.

\bibitem{7} Git Feature Branch Workflow, [Online] Available: https://www.atlassian.com/git/tutorials/comparing-workflows/feature-branch-workflow, [Accessed: Feb. 6, 2018].

  Each issue has an associated feature branch, and code necessary to develop the issue is commited to the appropriate feature branch.
  Then a pull request is made between the feature branch and development/master, and once all tests pass the feature branch can
  be merged to development/master.

\bibitem{8} Travis, "Travis homepage", [Online] Available: https://travis-ci.org/, [Accessed: Feb. 6, 2018].

  Travis is a popular continuos integration tool hosted typically on an external cloud. Free to use for open source
  projects.

\bibitem{9} Circle CI, "Circle CI  homepage", [Online] Available: https://circleci.com/, [Accessed: Feb. 6, 2018].

  Circle CI is very similar to Travis, but it allows a certain amount of free build hours per month for private
  repositories.

\bibitem{10} Jenkins, "Jenkins homepage", [Online] Available: https://jenkins.io/, [Accessed: Feb. 6, 2018].

  Jenkins is an industry proven automation tool more powerful than both Travis and Circle CI. It is typically deployed
  on premises as a Java web application and gives developers more control over the build agents compared to its
  competitors.

\bibitem{11} ZenHub, "ZenHub homepage", [Online] Available: https://www.zenhub.com/, [Accessed: Feb. 6, 2018].

    ZenHub is an agile project management chrome extension for GitHub. It provides the ability to assign points
    to issues, create epics, provides velocity tracing burndown charts for milestones, and a Kanban board to
    better manage work at hand.

\end{thebibliography}

% %
% % example of including an annotated bibliography. The current style is an author date one. If you want to change, comment out the line and uncomment the subsequent line. You should also modify the packages included at the top (see the notes earlier in the file) and then trash your aux files and re-run.
% %\bibliographystyle{authordate2annot}
% \bibliographystyle{IEEEannotU}
% \renewcommand{\refname}{Annotated Bibliography}  % if you put text into the final {} on this line, you will get an extra title, e.g. References. This isn't necessary for the outline project specification.
%
% \bibliography{mmp} % References file

\end{document}
