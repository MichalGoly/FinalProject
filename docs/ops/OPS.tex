\documentclass[11pt,fleqn,twoside]{article}
\usepackage{makeidx}
\makeindex
\usepackage{palatino} %or {times} etc
\usepackage{plain} %bibliography style
\usepackage{amsmath} %math fonts - just in case
\usepackage{amsfonts} %math fonts
\usepackage{amssymb} %math fonts
\usepackage{lastpage} %for footer page numbers
\usepackage{fancyhdr} %header and footer package
\usepackage{mmpv2}
%\usepackage{url}
\usepackage{hyperref}

% the following packages are used for citations - You only need to include one.
%
% Use the cite package if you are using the numeric style (e.g. IEEEannot).
% Use the natbib package if you are using the author-date style (e.g. authordate2annot).
% Only use one of these and comment out the other one.
\usepackage{cite}
%\usepackage{natbib}

\begin{document}

\name{Michal Goly}
\userid{mwg2}
\projecttitle{Quiz Tool}
\projecttitlememoir{Quiz Tool} %same as the project title or abridged version for page header
\reporttitle{Outline Project Specification}
\version{0.1}
\docstatus{Draft}
\modulecode{CS39440}
\degreeschemecode{G600}
\degreeschemename{Software Engineering}
\supervisor{Chris Loftus} % e.g. Neil Taylor
\supervisorid{cwl}
\wordcount{}

%optional - comment out next line to use current date for the document
%\documentdate{10th February 2014}
\mmp

\setcounter{tocdepth}{3} %set required number of level in table of contents


%==============================================================================
\section{Project description}
%==============================================================================
Student engagement and live knowledge monitoring are vital in the provision of good quality
lecture content in 2018. It is important to make sure audience understands concepts presented,
and quizes can help lecturers judge students' understanding in real time.

Aberystwyth University currently uses Qwizdom live polling tool during the provision
of some lectures and practical sessions. The university operates under a single license
forcing lecturers to book sessions before they can use the tool. Due to human nature,
session hijacking occasionally occurs to the bemusement of both students and lecturers.
For example, students could be shown biology slides half way through their geography
lecture.

This project will focus on the design and development of an in-house built Quiz Tool,
enabling multiple lecturers to use it at the same time and potentially making Qwizdom
redundant in the future. This ambition can only be achieved if the project is of
high quality and its future maintainability is considered at all stages of the design
and development.

The Quiz Tool will allow lecturers to login with their university credentials and
they will be authenticated using the university LDAP. Lecturers will then
be able to upload their PDF lecture slides and create \textit{Lectures} in the system.
Each \textit{Lecture} can be then edited and quizes can be added between slides.
An example of a quiz would be an "ABC style" question where a student can select
one of the options and submit their answer. Once a lecturer is happy with their
\textit{Lecture}, he can broadcast it and receive a session key which can be
shared with students. Lecture slides will be shown to all students and the
lecture can be delivered in a traditional fashion up to the moment a quiz
has been embedded beteen two slides. Students will then be able to answer
the question and polling results will be presented in real time to a lecturer.
It will be also possible to export the quiz answers as JSON for future analysis.

The tool will be composed of the back-end with an associated database, and two
front ends. One for lecturers and one for students. Finally, the tool will be
developed using an agile methodology, adjusted for a single person project.

%==============================================================================
\section{Proposed tasks}
%==============================================================================
1. Running within the aber intranet or not
2. External cloud provider vs local debian container
3. Authentication concerns -> how to access LDAP from the outside of the university network
4. Technology used
5. Angular 4 investigation
6. Real time nature of the project
7. Continuos delivery
8. Setting up the version control with suitable safe-guards
9. Agile methodology
10. Use of Docker
11. docker-compose vs kubernetes
12. Persistence and backups of the MongoDB container

%==============================================================================
\section{Project deliverables}
%==============================================================================
1. Containerised back-end consisting of an Express app with a MongoDB persistence layer
2. Containerised front-end in Angular 4 for lecturers and students
3. Final project

% the following line is included so that the bibliography is also shown in the table of contents. There is the possibility that this is added to the previous page for the bibliography. To address this, a newline is added so that it appears on the first page for the bibliography.
\newpage
\addcontentsline{toc}{section}{Initial Annotated Bibliography}

%
% example of including an annotated bibliography. The current style is an author date one. If you want to change, comment out the line and uncomment the subsequent line. You should also modify the packages included at the top (see the notes earlier in the file) and then trash your aux files and re-run.
%\bibliographystyle{authordate2annot}
\bibliographystyle{IEEEannotU}
\renewcommand{\refname}{Annotated Bibliography}  % if you put text into the final {} on this line, you will get an extra title, e.g. References. This isn't necessary for the outline project specification.
\bibliography{mmp} % References file

\end{document}
