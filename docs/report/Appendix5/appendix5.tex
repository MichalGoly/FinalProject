\chapter{Code Examples}
\label{chap:codesamples}

This appendix contains code examples relevant to the discussion in the main body of the
report.

\newpage
\section{Initial Front End Dockerfile}
The initial \texttt{Dockerfile} of the client container added in the first sprint of
the development of the tool.

\begin{figure}[ht]
  \begin{lstlisting}[basicstyle=\small, breaklines=true]
    ### STAGE 1: Build ###

    # We label our stage as 'builder'
    FROM node:8-alpine as builder

    COPY package.json package-lock.json ./

    RUN npm set progress=false && npm config set depth 0 && npm cache clean --force

    ## Storing node modules on a separate layer will prevent unnecessary npm installs at each build
    RUN npm i && mkdir /ng-app && cp -R ./node_modules ./ng-app

    WORKDIR /ng-app

    COPY . .

    ## Build the angular app in production mode and store the artifacts in dist folder
    RUN $(npm bin)/ng build --prod --build-optimizer

    ### STAGE 2: Setup ###

    FROM nginx:1.13.3-alpine

    ## Copy our default nginx config
    COPY nginx/default.conf /etc/nginx/conf.d/

    ## Remove default nginx website
    RUN rm -rf /usr/share/nginx/html/*

    ## From 'builder' stage copy over the artifacts in dist folder to default nginx public folder
    COPY --from=builder /ng-app/dist /usr/share/nginx/html

    CMD ["nginx", "-g", "daemon off;"]
  \end{lstlisting}
  \caption{Initial Front End Dockerfile}
  \label{sample:clientdocker}
\end{figure}

\newpage
\section{Initial Back End Dockerfile}
The initial \texttt{Dockerfile} of the server\_node container added in the first sprint of
the development of the tool.

\begin{figure}[h!]
    \begin{lstlisting}[basicstyle=\small]
    FROM node:carbon
    WORKDIR /usr/src/app
    COPY package*.json ./
    RUN npm install
    COPY . .
    EXPOSE 3000
    CMD [ "npm", "start" ]
  \end{lstlisting}
  \caption{Back End Dockerfile}
  \label{sample:serverdocker}
\end{figure}

\newpage
\section{Initial Circle CI Config File}
The initial \texttt{.circleci/config.yml} added in the first sprint of the development.

\begin{figure}[h!]
  \begin{lstlisting}[basicstyle=\small, breaklines=true]
  version: 2
  jobs:
    build:
      machine: true

      working_directory: ~/repo

      steps:
        - checkout

        - run:
            name: install docker-compose
            command: |
              set -x
              sudo chown -R $(whoami) /usr/local/bin
              curl -L https://github.com/docker/compose/releases/download/1.11.2/docker-compose-`uname -s`-`uname -m` > /usr/local/bin/docker-compose
              chmod +x /usr/local/bin/docker-compose

        - run:
            name: docker compose build image
            command: |
              set -x
              docker-compose build
              docker-compose up

        - run:
            name: unit tests
            command: |
              curl localhost

        # deploy only master branch
        - deploy:
            command: |
              if [ "${CIRCLE_BRANCH}" == "master" ]; then
                chmod +x scripts/deploy.sh
                ./scripts/deploy.sh
              fi
  \end{lstlisting}
  \caption{Initial Build Config File}
  \label{sample:circleci}
\end{figure}

\newpage
\section{Initial Dockerrun.aws.json file}
The initial version of the file specifying how containers should be linked together when running
in the production environment on AWS.

\begin{figure}[h!]
  \begin{lstlisting}[basicstyle=\small, breaklines=true]
  {
      "AWSEBDockerrunVersion": 2,
      "containerDefinitions": [
          {
              "name": "client",
              "image": "993389244112.dkr.ecr.eu-west-2.amazonaws.com/quiz-tool-client:latest",
              "memory": 128,
              "essential": true,
              "portMappings": [
                  {
                      "hostPort": 80,
                      "containerPort": 80
                  }
              ],
              "links": [
                  "server_node"
              ]
          },
          {
              "name": "server_node",
              "image": "993389244112.dkr.ecr.eu-west-2.amazonaws.com/quiz-tool-server:latest",
              "memory": 128,
              "essential": true,
              "links": [
                "database"
              ]
          },
          {
            "name": "database",
            "image": "mongo",
            "memory": 128,
            "essential": true,
            "portMappings": [
                {
                    "hostPort": 27017,
                    "containerPort": 27017
                }
            ]
          }
      ]
  }
  \end{lstlisting}
  \caption{Dockerrun.aws.json}
  \label{sample:dockerrunaws}
\end{figure}

\newpage
\section{Production Deployment Script}
The bash script performing the deployment from Circle CI to the production environment
hosted on AWS.

\begin{figure}[h!]
  \begin{lstlisting}[basicstyle=\small, breaklines=true]
  #!/bin/bash
  set -x
  set -e

  AWS_CONFIG_FILE=$HOME/.aws/config

  mkdir $HOME/.aws
  touch $AWS_CONFIG_FILE
  chmod 600 $AWS_CONFIG_FILE

  echo "[default]"                                     > $AWS_CONFIG_FILE
  echo "aws_access_key_id=$AWS_ACCESS_KEY_ID"         >> $AWS_CONFIG_FILE
  echo "aws_secret_access_key=$AWS_SECRET_ACCESS_KEY" >> $AWS_CONFIG_FILE

  $(aws ecr get-login --no-include-email --region eu-west-2)

  docker tag repo_client:latest 993389244112.dkr.ecr.eu-west-2.amazonaws.com/quiz-tool-client:latest
  docker tag repo_server_node:latest 993389244112.dkr.ecr.eu-west-2.amazonaws.com/quiz-tool-server:latest

  docker push 993389244112.dkr.ecr.eu-west-2.amazonaws.com/quiz-tool-client:latest
  docker push 993389244112.dkr.ecr.eu-west-2.amazonaws.com/quiz-tool-server:latest

  eb deploy prod-env
  \end{lstlisting}
  \caption{Production Deployment Script}
  \label{sample:deploy}
\end{figure}

\newpage
\section{Final Dockerrun.aws.json file}
The final version of the file specifying how Docker containers should be link with each other
when running in the production environment provided by AWS.
\begin{figure}[h!]
  \begin{lstlisting}[basicstyle=\tiny, breaklines=true]
{
    "AWSEBDockerrunVersion": 2,
    "volumes": [
      {
        "name": "mongo",
        "host": {
          "sourcePath": "/var/app/database"
        }
      }
    ],
    "containerDefinitions": [
        {
            "name": "client",
            "image": "993389244112.dkr.ecr.eu-west-2.amazonaws.com/quiz-tool-client:latest",
            "memory": 300,
            "essential": true,
            "portMappings": [
                {
                    "hostPort": 80,
                    "containerPort": 80
                }
            ],
            "links": [
                "server_node"
            ]
        },
        {
            "name": "server_node",
            "image": "993389244112.dkr.ecr.eu-west-2.amazonaws.com/quiz-tool-server:latest",
            "memory": 300,
            "essential": true,
            "links": [
              "database"
            ]
        },
        {
          "name": "database",
          "image": "mongo",
          "memory": 300,
          "essential": true,
          "mountPoints": [
              {
                "sourceVolume": "mongo",
                "containerPath": "/data/db"
              }
          ],
          "portMappings": [
              {
                  "hostPort": 27017,
                  "containerPort": 27017
              }
          ]
        }
    ]
}
\end{lstlisting}
\caption{Final Dockerrun.aws.json}
\label{sample:dockerrunawsfinal}
\end{figure}

\newpage
\section{Final Back End Dockerfile}
The final \texttt{Dockerfile} of the server\_node container.

\begin{figure}[h!]
    \begin{lstlisting}[basicstyle=\small, breaklines=true]
    FROM node:carbon

    # Create app directory
    WORKDIR /usr/src/app

    # Install app dependencies
    # A wildcard is used to ensure both package.json AND package-lock.json are copied
    # where available (npm@5+)
    COPY package*.json ./

    # Install system dependencies
    RUN apt-get update && apt-get install -y --no-install-recommends \
    		ghostscript \
    		pdftk \
    		poppler-utils \
    		graphicsmagick

    RUN npm install
    # If you are building your code for production
    # RUN npm install --only=production

    # Bundle app source
    COPY . .

    EXPOSE 3000
    CMD [ "npm", "start" ]
  \end{lstlisting}
  \caption{Back End Dockerfile}
  \label{sample:finalserverdocker}
\end{figure}

\newpage
\section{Controller Example}
The final implementation of the \texttt{slides.controller.js}.

\begin{figure}[h!]
  \begin{lstlisting}[basicstyle=\tiny, breaklines=true]
  var express = require('express');
  var router = express.Router();
  var authHelper = require('../helpers/auth.helper');
  var lecturesDb = require('../db/lectures.db');
  var slidesDb = require('../db/slides.db');
  var Slide = require('../models/slide');
  var Validator = require('jsonschema').Validator;
  var validator = new Validator();
  var BulkSlideUpdateSchema = require('../schemas/bulkSlideUpdate');

  router.get('/:lecture_id', getSlides);
  router.put('/', bulkUpdateQuiz);

  module.exports = router;

  function getSlides(req, res) {
    authHelper.check(req, res).then((lecturer) => {
      lecturesDb.getOne(req.params.lecture_id).then((lecture) => {
        if (lecturer._id == lecture.lecturerId) {
          slidesDb.getByLectureId(lecture._id).then((slides) => {
            var out = [];
            for (var i = 0; i < slides.length; i++) {
              out.push({
                _id: slides[i]._id,
                lectureId: slides[i].lectureId,
                image: slides[i].image.toString('base64'),
                text: slides[i].text,
                quizType: slides[i].quizType,
                slideNumber: slides[i].slideNumber
              });
            }
            res.send(out);
          }).catch((err) => {
            console.error("An error has occurred: " + err);
            res.status(500).send({
              error: err
            });
          });
        } else {
          res.status(401).send({
            error: "Unauthorized"
          });
        }
      }).catch((err) => {
        if (err === 400) {
          res.status(400).send({
            error: "Bad request, lectureId malformed"
          });
        } else {
          console.error("An error has occurred: " + err);
          res.status(500).send({
            error: err
          });
        }
      });
    }).catch((err) => {
      res.status(401).send({
        error: "Unauthorized"
      });
    });
  }

  // Bulk updates quizType of the slides provided
  function bulkUpdateQuiz(req, res) {
    if (validator.validate(req.body, BulkSlideUpdateSchema).valid) {
      authHelper.check(req, res).then((lecturer) => {
        slidesDb.bulkUpdateQuiz(req.body).then(() => {
          res.sendStatus(200);
        }).catch((err) => {
          console.error("An error has occurred: " + err);
          res.status(500).send({
            error: "Internal server error"
          });
        });
      }).catch((err) => {
        res.status(401).send({
          error: "Unauthorized"
        });
      });
    } else {
      res.status(400).send({
        error: "Bad request"
      });
    }
  }
  \end{lstlisting}
  \caption{Example of a controller implementation}
  \label{sample:controller}
\end{figure}

\newpage
\section{Final Nginx Config File}
The final \texttt{default.conf} nginx configuration file, enabling routing within the
application and preventing people from uploading files over 15 MB.

\begin{figure}[h!]
  \begin{lstlisting}[basicstyle=\tiny, breaklines=true]
  server
  {
    listen 80;
    sendfile on;
    default_type application/octet-stream;

    gzip on;
    gzip_http_version 1.1;
    gzip_disable "MSIE [1-6]\.";
    gzip_min_length 256;
    gzip_vary on;
    gzip_proxied expired no-cache no-store private auth;
    gzip_types text/plain text/css application/json application/javascript application/x-javascript text/xml application/xml application/xml+rss text/javascript;
    gzip_comp_level 9;

    root /usr/share/nginx/html;

    location /
    {
      try_files $uri $uri/ /index.html =404;
    }

    location /express/
    {
      proxy_set_header X-Real-IP $remote_addr;
      proxy_set_header X-Forwarded-For $proxy_add_x_forwarded_for;
      proxy_set_header Host $http_host;
      proxy_set_header X-NginX-Proxy true;

      proxy_pass http://server_node:3000/;

      proxy_http_version 1.1;
      proxy_set_header Upgrade $http_upgrade;
      proxy_set_header Connection "upgrade";

      client_max_body_size 15M;
    }

    location ~* \.io
    {
      proxy_set_header X-Real-IP $remote_addr;
      proxy_set_header X-Forwarded-For $proxy_add_x_forwarded_for;
      proxy_set_header Host $http_host;
      proxy_set_header X-NginX-Proxy true;

      proxy_pass http://server_node:3000;
      proxy_redirect off;

      proxy_http_version 1.1;
      proxy_set_header Upgrade $http_upgrade;
      proxy_set_header Connection "upgrade";
    }

  }
  \end{lstlisting}
  \caption{Final Nginx Config File}
  \label{sample:nginx}
\end{figure}

\newpage
\section{Final Circle CI Config File}
The final \texttt{.circleci/config.yml} which evolved over the iterations.

\begin{figure}[h!]
  \begin{lstlisting}[basicstyle=\tiny, breaklines=true]
  # Javascript Node CircleCI 2.0 configuration file
  #
  # Check https://circleci.com/docs/2.0/language-javascript/ for more details
  #
  version: 2
  jobs:
    build:
      machine: true

      working_directory: ~/repo

      steps:
        - checkout

        - run:
            name: AWS integration
            command: |
              sudo apt-get -y -qq update
              sudo apt-get install python-pip python-dev build-essential
              sudo pip install awsebcli --upgrade
              eb --version

        - run:
            name: install docker-compose
            command: |
              set -x
              sudo chown -R $(whoami) /usr/local/bin
              curl -L https://github.com/docker/compose/releases/download/1.16.1/docker-compose-`uname -s`-`uname -m` > /usr/local/bin/docker-compose
              chmod +x /usr/local/bin/docker-compose

        - run:
            name: docker compose build image
            command: |
              touch .env
              set -x
              docker-compose build

        - run:
            name: run client unit tests
            command: |
              cd client
              docker build -t clienttest --file Dockerfile.test .
              docker run -it clienttest
              cd ..

        - run:
            name: run server unit tests
            command: |
              echo "GOOGLE_CLIENT_ID=$GOOGLE_CLIENT_ID" >> .env
              echo "GOOGLE_CLIENT_SECRET=$GOOGLE_CLIENT_SECRET" >> .env
              echo "GOOGLE_CALLBACK_URL=$GOOGLE_CALLBACK_URL" >> .env
              docker-compose -f docker-compose.test.yml up --abort-on-container-exit --exit-code-from server_node
              docker-compose -f docker-compose.test.yml down

        - run:
            name: run integration tests
            command: |
              docker-compose -f docker-compose.integration.yml build
              docker-compose -f docker-compose.integration.yml up --abort-on-container-exit --exit-code-from tester
              docker-compose -f docker-compose.integration.yml down

        # deploy only master and develop branch
        - deploy:
            command: |
              if [ "${CIRCLE_BRANCH}" == "master" ]; then
                chmod +x scripts/deploy.sh
                ./scripts/deploy.sh
              fi
  \end{lstlisting}
  \caption{Final Build Config File}
  \label{sample:circlecifinal}
\end{figure}

\newpage
\section{Front End Dockerfile for Testing}
\texttt{Dockerfile.test} used for front end unit tests to address the lack of \texttt{npm}
in the production container shown in the \autoref{sample:clientdocker}.

\begin{figure}[ht]
  \begin{lstlisting}[basicstyle=\small, breaklines=true]
  # Dockerfile for client side unit tests
  FROM node:8

  COPY package.json package-lock.json ./

  RUN npm set progress=false && npm config set depth 0 && npm cache clean --force

  ## Storing node modules on a separate layer will prevent unnecessary npm installs at each build
  RUN npm i && mkdir /ng-app && cp -R ./node_modules ./ng-app

  WORKDIR /ng-app

  COPY . .

  RUN $(npm bin)/ng build

  CMD npm test
  \end{lstlisting}
  \caption{Dockerfile.test Used For Front End Testing}
  \label{sample:clientdockertest}
\end{figure}

\newpage
\section{Integration Testing Docker Environment}
\texttt{docker-compose.integration.yml} file added to define how containers should be
linked together for selenium integration testing.

\begin{figure}[ht]
  \begin{lstlisting}[basicstyle=\tiny, breaklines=true]
  version: '2.0'
  services:
    # Quiz Tool to be tested
    client:
      build: client
      ports:
        - "80:80"
      links:
        - server_node
      depends_on:
        - chrome
        - firefox
    server_node:
      build: server
      environment:
        - NODE_ENV=test
        - AUTH_DISABLED=true
        - GOOGLE_AUTH_DISABLED=true
      env_file:
        - .env
      links:
        - database
      depends_on:
        - chrome
        - firefox
    database:
      image: mongo
      ports:
        - "27017:27017"
      depends_on:
        - chrome
        - firefox
    # Integration tests runner
    tester:
      build: tester
      depends_on:
        - chrome
        - firefox
        - selenium-hub
      env_file:
        - .env
      entrypoint: npm test
    # Selenium hub for integration testing
    selenium-hub:
      image: selenium/hub
      container_name: selenium-hub
      ports:
        - "4444:4444"
      environment:
        - GRID_MAX_SESSION=10
    chrome:
      image: selenium/node-chrome
      depends_on:
        - selenium-hub
      volumes:
        - /dev/shm:/dev/shm
      environment:
        - HUB_PORT_4444_TCP_ADDR=selenium-hub
        - HUB_PORT_4444_TCP_PORT=4444
        - NODE_MAX_SESSION=10
        - NODE_MAX_INSTANCES=10
    firefox:
      image: selenium/node-firefox
      depends_on:
        - selenium-hub
      volumes:
        - /dev/shm:/dev/shm
      environment:
        - HUB_PORT_4444_TCP_ADDR=selenium-hub
        - HUB_PORT_4444_TCP_PORT=4444
        - NODE_MAX_SESSION=10
        - NODE_MAX_INSTANCES=10
  \end{lstlisting}
  \caption{docker-compose.integration.yml Used For Selenium Testing}
  \label{sample:dockercomposeintegration}
\end{figure}
