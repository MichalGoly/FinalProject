\thispagestyle{empty}

\begin{center}
    {\LARGE\bf Abstract}
\end{center}
Student engagement and live knowledge monitoring are vital in the provision of good quality
lecture content in 2018. It is important to make sure audience understands concepts presented,
and quizzes can help lecturers judge students' understanding in real time.

Aberystwyth University currently uses Qwizdom\cite{1} live polling tool during the provision
of some lectures and practical sessions. The university operates under a single license
forcing lecturers to book sessions before they can use the tool. Due to human nature,
session hijacking occasionally occurs to the bemusement of both students and lecturers.
For example, students could be shown biology slides half way through their geography
lecture.

This project focused on the design and development of an in-house built Quiz Tool,
enabling multiple lecturers to use it at the same time and potentially making Qwizdom
redundant in the future. This ambition could only be achieved if the project was of
high quality and its future maintainability was considered at all stages of the design
and development.

The Quiz Tool allows lecturers to login using their Google Single Sign-on\cite{2} credentials,
upload their PDF lecture slides and create \textit{Lectures} in the system.
Each \textit{Lecture} can be then edited, and eligible slides can be marked as quizzes.
True/false, single and multi choice style quizzes are supported. Once a lecturer is happy with their
\textit{Lecture}, he can broadcast it and receive a session key which can be
shared with students. Lecture slides will be shown to all students and the
lecture can be delivered in a traditional fashion up to the moment a slide has been marked as a quiz.
Students will then be able to answer
the question and polling results will be presented in real time to the lecturer.
Lecture sessions broadcasted in the past are kept, and students' answers can be exported
as a PDF report for future analysis.

The tool is composed of a back-end with an associated database, and two
front-ends. One for lecturers and one for students. Finally, the tool has been
successfully developed using an agile methodology, adjusted for a single person
project.
